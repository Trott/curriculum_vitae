%% start of file `cv.tex'.
%% Rich Trott CV ~ Oct 2015
%%
%% Adapted from moderncv template.tex
%% Copyright 2006-2012 Xavier Danaux (xdanaux@gmail.com).
%
% This work may be distributed and/or modified under the
% conditions of the LaTeX Project Public License version 1.3c,
% available at http://www.latex-project.org/lppl/.


\documentclass[10pt,letterpaper,sans]{moderncv}   % possible options include font size ('10pt', '11pt' and '12pt'), paper size ('a4paper', 'letterpaper', 'a5paper', 'legalpaper', 'executivepaper' and 'landscape') and font family ('sans' and 'roman')


% moderncv themes
\moderncvstyle{banking}                        % style options are 'casual' (default), 'classic', 'oldstyle' and 'banking'
\moderncvcolor{black}                          % color options 'blue' (default), 'orange', 'green', 'red', 'purple', 'grey' and 'black'

% adjust the page margins
\usepackage[scale=0.86]{geometry}

% personal data
\firstname{Rich}
\familyname{Trott}
% \address{123 Fake St.}{Stanford, California 94305}

\mobile{+1~(415)~596~0558}
\email{r.trott@gmail.com}
\homepage{trott.github.io}
% \extrainfo{www.github.com/trott}

%----------------------------------------------------------------------------------
%            content
%----------------------------------------------------------------------------------
\begin{document}
%-----       resume       ---------------------------------------------------------
\makecvtitle

\section{Experience}

\cventry{January 2012 - Present}{Director, Center for Knowledge Management}{UC San Francisco}{San Francisco, California}{}{I led a team of approximately 12 software engineers on a wide variety of projects. I also acted as an individual contributor on many projects. During my tenure, the team transitioned to a default-to-open policy and subsequent contributions to open source greatly increased. Projects were in areas that included open data and open access, health sciences education, web, and mobile.\\}

\cventry{2008 - 2012}{Applications Lead}{UC San Francisco}{}{}{My role during this period was similar to the role role described above except that my role was limited to software engineering and did not include infrastructure and user experience.\\}

\cventry{2007 - 2008}{Security Analyst}{UC San Francisco}{}{}{My role included both a technical component and a policy component. I created and maintained security tools, evaluated emerging security and compliance technology, and performed institution-wide and application-specific risk assessments.\\}

\cventry{1999 - 2007}{Programmer/Analyst}{UC San Francisco}{}{}{This job covered several roles over time, all performed under one job title. I started as a systems administrator, moved to programming, and was leading a small team of programmers at the end.\\}

\cventry{1998-1999}{Web Administrator and Developer}{Autodesk Inc.}{San Rafael, California}{}{Migrated website architecture from a pile of Perl CGIs to an early content management system.\\}

\cventry{1997-1998}{Systems Administrator}{LucasArts Entertainment Company, LLC}{San Rafael, California}{}{I worked on server farms and firewalls and pretty much anything else that needed to be worked on.\\}

\cventry{1996-1997}{Systems Programmer}{Rutgers University}{New Brunswick, New Jersey}{}{I dropped out of grad school and took a "real" programming job. Ten minutes later, everyone was moving to Silicon Valley to work in tech.\\}

\cventry{1993-1996}{Campus Computing Facilities Manager}{Rutgers University}{New Brunswick, New Jersey}{}{Although I was ostensibly managing student computing facilities, I found lots of excuses to write code and experiment with the then-emerging web.\\}

\cventry{1991-1993}{Programmer}{AT\&T}{Basking Ridge, New Jersey}{}{I worked with COBOL and JCL on reporting tasks that would be done today by an eighth grader using Excel. I spent most of the time looking longingly at the other end of the building where the UNIX and C programmers were working.\\}

% \section{Education}
% \cventry{2012--2014}{MA/MST}{Stanford University}{Stanford, California}{\textit{Masters of Music Science and Technology}}{Two year masters program at the \textit{Center for Computer Research in Music and Acoustics}.  Studies include human computer interaction design, digital signal processing, systems design, computer music composition, and site-based installation art. \\\textit{Recipient of the Denning Family Fellowship in Fine Arts }}
% \cventry{2009--2012}{BFA}{OCAD University}{Toronto, Ontario}{\textit{Integrated Media with minor in Digital Media Studies}}{Graduated on the Dean's honor list. Studies primarily focused on interactive audio/visual site-based installation art.  Other course work included physical computing, computer science, media theory, cultural rhetoric, and rapid-prototyping / manufacturing. \\\textit{Recipient of the OCAD University Medal for Integrated Media }}

% \section{Awards}
% \cventry{}{OCAD University}{OCAD University Medal in Integrated Media}{May 2012}{}{Top accolade given to one student from each department at time of graduation}
% \cventry{}{Stanford Arts Institute}{Denning Family Fellowship in Fine Arts
% }{March 2012}{}{}
% \cventry{}{OCAD University}{Project 31 Integrated Media Faculty Scholarship
% }{May 2011}{}{}
% \cventry{}{OCAD University}{DFI Award}{May 2010}{}{}
% \cventry{}{OCAD University}{InterAccess Media Prize}{May 2010}{}{}

% \cventry{2010--2012}{Research Assistant}{OCAD University}{Toronto, Ontario}{}{Developed code and interaction design frameworks for various government funded research projects including "Body Editing" and "Bio Mapping" with Paula Gardiner, and "Haptics" with Michael Page.}

\section{Skills}
\cvitem{Programming (fluent/proficient)}{JavaScript, PHP, Perl, bash}
\cvitem{Programming (familiar)}{C, C++, Java, Ruby, Python, Objective-C}
\cvitem{Web Technologies}{JavaScript, HTML5, CSS3, assorted tooling and preprocessors, browserify, npm, bower}
% \cvitem{Prototyping}{3D Printing, CNC-Milling, Laser Cutting, Wood/Metal/Plastics shop experience}
\cvitem{Project Contributor}{Notably Node.js; authored or contributed to many other projects}

\end{document}
